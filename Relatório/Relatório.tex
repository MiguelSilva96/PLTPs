%
% Layout retirado de http://www.di.uminho.pt/~prh/curplc09.html#notas
%
\documentclass{report}
\usepackage[portuges]{babel}
\usepackage[utf8]{inputenc}
%\usepackage[latin1]{inputenc}

\usepackage{url}
\usepackage{enumerate}
\usepackage{graphicx}
\graphicspath{ {testImages/} }

%\usepackage{alltt}
%\usepackage{fancyvrb}
\usepackage{listings}
\usepackage{eurosym}
%LISTING - GENERAL
\lstset{
    language=Awk,
    basicstyle=\ttfamily\small,
    numberstyle=\footnotesize,
    numbers=left,
    frame=single,
    tabsize=2,
    title=\lstname,
    escapeinside={\%*}{*)},
    breaklines=true,
    breakatwhitespace=true,
    framextopmargin=2pt,
    framexbottommargin=2pt,
    inputencoding=utf8,
    extendedchars=true,
    showspaces=false,
    showstringspaces=false,
    literate={á}{{\'a}}1 {é}{{\'e}}1 {í}{{\'i}}1 {ó}{{\'o}}1 {ú}{{\'u}}1
    {Á}{{\'A}}1 {É}{{\'E}}1 {Í}{{\'I}}1 {Ó}{{\'O}}1 {Ú}{{\'U}}1
    {à}{{\`a}}1 {è}{{\`e}}1 {ì}{{\`i}}1 {ò}{{\`o}}1 {ù}{{\`u}}1
    {À}{{\`A}}1 {È}{{\'E}}1 {Ì}{{\`I}}1 {Ò}{{\`O}}1 {Ù}{{\`U}}1
    {â}{{\^a}}1 {ê}{{\^e}}1 {î}{{\^i}}1 {ô}{{\^o}}1 {û}{{\^u}}1
    {ã}{{\~a}}1 {º}{{\textsuperscript{o}}}1 {ç}{{\c c}}1 {Ç}{{\c C}}1
    {€}{{\euro}}1
}


%
%\lstset{ %
%	language=Java,							% choose the language of the code
%	basicstyle=\ttfamily\footnotesize,		% the size of the fonts that are used for the code
%	keywordstyle=\bfseries,					% set the keyword style
%	%numbers=left,							% where to put the line-numbers
%	numberstyle=\scriptsize,				% the size of the fonts that are used for the line-numbers
%	stepnumber=2,							% the step between two line-numbers. If it's 1 each line
%											% will be numbered
%	numbersep=5pt,							% how far the line-numbers are from the code
%	backgroundcolor=\color{white},			% choose the background color. You must add \usepackage{color}
%	showspaces=false,						% show spaces adding particular underscores
%	showstringspaces=false,					% underline spaces within strings
%	showtabs=false,							% show tabs within strings adding particular underscores
%	frame=none,								% adds a frame around the code
%	%abovecaptionskip=-.8em,
%	%belowcaptionskip=.7em,
%	tabsize=2,								% sets default tabsize to 2 spaces
%	captionpos=b,							% sets the caption-position to bottom
%	breaklines=true,						% sets automatic line breaking
%	breakatwhitespace=false,				% sets if automatic breaks should only happen at whitespace
%	title=\lstname,							% show the filename of files included with \lstinputlisting;
%											% also try caption instead of title
%	escapeinside={\%*}{*)},					% if you want to add a comment within your code
%	morekeywords={*,...}					% if you want to add more keywords to the set
%}

\usepackage{xspace}

\parindent=0pt
\parskip=2pt

\setlength{\oddsidemargin}{-1cm}
\setlength{\textwidth}{18cm}
\setlength{\headsep}{-1cm}
\setlength{\textheight}{23cm}

\def\pt{\emph{Processador de Texto}\xspace}
\def\fs{\emph{Field Seperator}\xspace}


\def\titulo#1{\section{#1}}
\def\super#1{{\em Supervisor: #1}\\ }
\def\area#1{{\em \'{A}rea: #1}\\[0.2cm]}
\def\resumo{\underline{Resumo}:\\ }


%%%%\input{LPgeneralDefintions}

\title{ \textbf{Normalizador de Autores em BibTex}\\ 
\textbf{\&} \\
\textbf{Processador de Inglês corrente} \\ 
\textbf{em Flex} \\ \textbf{} \\
Processamento de Linguagens\\(3º ano de Curso)\\ 
\textbf{Trabalho Prático nº1 - Parte B}\\ Relatório de Desenvolvimento}
\author{José Silva\\ (A74601) \and Pedro Cunha\\ (A73958) \and Gonçalo Moreira\\ (A73591) }
\date{\today}

\begin{document}

\maketitle

\begin{abstract}
Documentação do segundo trabalho prático da unidade curricular de "Processamento de Linguagens", o principal foco incide em utilizar o analisador léxico FLEX, para desenvolver processadores de texto. 
Simplifica o trabalho que seria necessário utilizando diretamente a linguagem de programação C, facilitando o processo. 
Demonstrando e documentando as soluções propostas pelo grupo de trabalho para os doius problemas escolhidos, termina-se o relatório com uma análise argumentativa sobre a eficiência dessas mesmas soluções. 
\end{abstract}

\tableofcontents


\chapter{Introdução} \label{intro}

Cenas introdutórias............
....................
..................
..............


\section*{Estrutura do Relatório}
No capítulo 1 faz-se uma pequena introdução ao problema e às ferramentas utilizadas para a resolução deste. Para além disso, é descrita de uma forma breve a estrutura do relatório.\par
No capítulo 2 faz-se uma análise breve mas mais detalhada do problema escolhido pelo grupo de trabalho.\par
No capítulo 3 é descrito de uma forma sumarizada o procedimento utilizado para solucionar as várias questões propostas pelos enunciados.\par
No capítulo 4 são apresentados alguns testes e respetivos resultados para comprovar o respectivo funcionamento das soluções apresentadas.\par
Finalmente, no capítulo 5 termina-se o relatório com uma síntese do que foi dito, as conclusões e o trabalho futuro.

\chapter{Análise e Especificação} \label{ae}

\section{Normalizador de Autores em BibTex}

\subsection{Descrição informal do problema}
É fornecido um ficheiro em BibTex(como input), com várias entradas e diversos autores e editores por entrada.
Pretende-se que se desenvolva um "Normalizador" para ler esse mesmo ficheiro e gerar um ficheiro equivalente
(em BibTex)com os nomes de todos os editores e autores normalizados. Também se pretende converter todos os carateres com acentos explícitos em caracteres portugueses.
A forma de normalização dos nomes e dos acentos explícitos é apresentada em detalhe a baixo, na Especificação dos Requisitos.
\subsection{Especificação dos Requisitos}
\subsubsection{Dados}
Como já foi referido, é fornecido um ficheiro em BibTex com várias entradas e diversos autores e editores por entrada. Este ficheiro contém, em cada entrada, diversos fields que poderão ou não incluir os fields author e editor. Podem existir
diversos tipos de entrada, com número e tipos diferentes de fields.\par
Cada field pode ocupar uma ou mais linhas, dependendo da informação que o mesmo representa. Na maioria dos casos de editor
e author ocupa apenas uma, mas não é regra. Quando existem muitos autores e/ou editores é normal que sejam
necessárias mais linhas. Podem também existir acentos explícitos em qualquer field que tenha texto, incluindo nos nomes
de editores e autores.
\subsubsection{Pedidos}
Como primeiro requisito, é pedido que todos os acentos explícitos(e cedilhas) sejam convertidos para caracteres portugueses,
 exemplos:\par
Anast\textbackslash'acia ou Anast\{\textbackslash'a\}cia deve ser convertido em: Anastácia. \\
Gon\textbackslash c\{c\}alo ou Gon\{\textbackslash c\{c\}\}alo deve ser convertido em: Gonçalo. \\
Também é pedido que os nomes de autores e editores sejam normalizados, todos têm que apresentar a mesma forma.
A forma requisitada é a seguinte: "Apelido, N1. N2." em que N1 e N2 são possíveis nomes próprios e Apelido é,
obviamente, o apelido(ou apelidos em alguns casos). Salienta-se que também é necessário que sejam todos da forma
"author/editor = \{ ... \}", por isso casos que usem aspas ou o número errado de espaços também devem ser 
corrigidos. Exemplo de normalização: \\
author="Martini, Ricardo G. and Ara{\textbackslash’u}jo, Cristiana and Almeida, 
Jos{\textbackslash’e} Jo{\textbackslash~a}o and Henriques, Pedro" \\
Deve ficar: \\
author = \{Martini, R. G. and Araújo, C. and Almeida, J. J. and Henriques, P.\}


\section{Processador de Inglês corrente}

\subsection{Descrição informal do problema}
RTFCYGVHBJDNKAFIYADVFABONAODBANPDIN\par
cgvhjkbsdnkgdmf\par
adhv abfk asf\par
ashdbasopdnaspdnsad

\subsection{Especificação dos Requisitos}
\subsubsection{Dados}
aldnsapidnsa+odsad+safmapsnfsapfm\par
fhgaj kbjlnkfadfm\par
ajfvhabsojdnksadpsaadpin\par
ahsgjdvhbasinfpafp\par
asdafadfaa
\subsubsection{Pedidos}
xsaddsafafafasfa

\chapter{Concepção/desenho da Resolução} \label{cd}

\section{Normalizador de Autores em BibTex}
\subsection{Estruturas de Dados}
y
\subsection{Algoritmos}
x

\section{Processador de Inglês corrente}
\subsection{Estruturas de Dados}
y
\subsection{Algoritmos}
x

\chapter{Codificação e Testes} \label{ct}

\section{Normalizador de Autores em BibTex}

\subsection{Alternativas, Decisões e Problemas de Implementação}
xyz

\subsection{Testes realizados e Resultados}
xyzzzz

\subsection{O programa é executado com ...}

%\includegraphics{}

\subsection{Obtém-se o seguinte resultado}

\section{Processador de Inglês corrente}

\subsection{Alternativas, Decisões e Problemas de Implementação}
xyz

\subsection{Testes realizados e Resultados}
xyzzzz

\subsection{O programa é executado com ...}

%\includegraphics{}

\subsection{Obtém-se o seguinte resultado}

\chapter{Conclusão} \label{concl}
Espetacular


\appendix 
\chapter{Código do Programa}

Lista-se a seguir o código  do programa  que foi desenvolvido.

%\lstinputlisting{}%input de um ficheiro

\bibliographystyle{alpha}
\bibliography{relprojLayout}



\end{document}
